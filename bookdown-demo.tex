% Options for packages loaded elsewhere
\PassOptionsToPackage{unicode}{hyperref}
\PassOptionsToPackage{hyphens}{url}
%
\documentclass[
]{book}
\usepackage{amsmath,amssymb}
\usepackage{lmodern}
\usepackage{iftex}
\ifPDFTeX
  \usepackage[T1]{fontenc}
  \usepackage[utf8]{inputenc}
  \usepackage{textcomp} % provide euro and other symbols
\else % if luatex or xetex
  \usepackage{unicode-math}
  \defaultfontfeatures{Scale=MatchLowercase}
  \defaultfontfeatures[\rmfamily]{Ligatures=TeX,Scale=1}
\fi
% Use upquote if available, for straight quotes in verbatim environments
\IfFileExists{upquote.sty}{\usepackage{upquote}}{}
\IfFileExists{microtype.sty}{% use microtype if available
  \usepackage[]{microtype}
  \UseMicrotypeSet[protrusion]{basicmath} % disable protrusion for tt fonts
}{}
\makeatletter
\@ifundefined{KOMAClassName}{% if non-KOMA class
  \IfFileExists{parskip.sty}{%
    \usepackage{parskip}
  }{% else
    \setlength{\parindent}{0pt}
    \setlength{\parskip}{6pt plus 2pt minus 1pt}}
}{% if KOMA class
  \KOMAoptions{parskip=half}}
\makeatother
\usepackage{xcolor}
\IfFileExists{xurl.sty}{\usepackage{xurl}}{} % add URL line breaks if available
\IfFileExists{bookmark.sty}{\usepackage{bookmark}}{\usepackage{hyperref}}
\hypersetup{
  pdftitle={An Introduction to Modular Representation Theory},
  pdfauthor={Amritanshu Prasad},
  hidelinks,
  pdfcreator={LaTeX via pandoc}}
\urlstyle{same} % disable monospaced font for URLs
\usepackage{color}
\usepackage{fancyvrb}
\newcommand{\VerbBar}{|}
\newcommand{\VERB}{\Verb[commandchars=\\\{\}]}
\DefineVerbatimEnvironment{Highlighting}{Verbatim}{commandchars=\\\{\}}
% Add ',fontsize=\small' for more characters per line
\usepackage{framed}
\definecolor{shadecolor}{RGB}{248,248,248}
\newenvironment{Shaded}{\begin{snugshade}}{\end{snugshade}}
\newcommand{\AlertTok}[1]{\textcolor[rgb]{0.94,0.16,0.16}{#1}}
\newcommand{\AnnotationTok}[1]{\textcolor[rgb]{0.56,0.35,0.01}{\textbf{\textit{#1}}}}
\newcommand{\AttributeTok}[1]{\textcolor[rgb]{0.77,0.63,0.00}{#1}}
\newcommand{\BaseNTok}[1]{\textcolor[rgb]{0.00,0.00,0.81}{#1}}
\newcommand{\BuiltInTok}[1]{#1}
\newcommand{\CharTok}[1]{\textcolor[rgb]{0.31,0.60,0.02}{#1}}
\newcommand{\CommentTok}[1]{\textcolor[rgb]{0.56,0.35,0.01}{\textit{#1}}}
\newcommand{\CommentVarTok}[1]{\textcolor[rgb]{0.56,0.35,0.01}{\textbf{\textit{#1}}}}
\newcommand{\ConstantTok}[1]{\textcolor[rgb]{0.00,0.00,0.00}{#1}}
\newcommand{\ControlFlowTok}[1]{\textcolor[rgb]{0.13,0.29,0.53}{\textbf{#1}}}
\newcommand{\DataTypeTok}[1]{\textcolor[rgb]{0.13,0.29,0.53}{#1}}
\newcommand{\DecValTok}[1]{\textcolor[rgb]{0.00,0.00,0.81}{#1}}
\newcommand{\DocumentationTok}[1]{\textcolor[rgb]{0.56,0.35,0.01}{\textbf{\textit{#1}}}}
\newcommand{\ErrorTok}[1]{\textcolor[rgb]{0.64,0.00,0.00}{\textbf{#1}}}
\newcommand{\ExtensionTok}[1]{#1}
\newcommand{\FloatTok}[1]{\textcolor[rgb]{0.00,0.00,0.81}{#1}}
\newcommand{\FunctionTok}[1]{\textcolor[rgb]{0.00,0.00,0.00}{#1}}
\newcommand{\ImportTok}[1]{#1}
\newcommand{\InformationTok}[1]{\textcolor[rgb]{0.56,0.35,0.01}{\textbf{\textit{#1}}}}
\newcommand{\KeywordTok}[1]{\textcolor[rgb]{0.13,0.29,0.53}{\textbf{#1}}}
\newcommand{\NormalTok}[1]{#1}
\newcommand{\OperatorTok}[1]{\textcolor[rgb]{0.81,0.36,0.00}{\textbf{#1}}}
\newcommand{\OtherTok}[1]{\textcolor[rgb]{0.56,0.35,0.01}{#1}}
\newcommand{\PreprocessorTok}[1]{\textcolor[rgb]{0.56,0.35,0.01}{\textit{#1}}}
\newcommand{\RegionMarkerTok}[1]{#1}
\newcommand{\SpecialCharTok}[1]{\textcolor[rgb]{0.00,0.00,0.00}{#1}}
\newcommand{\SpecialStringTok}[1]{\textcolor[rgb]{0.31,0.60,0.02}{#1}}
\newcommand{\StringTok}[1]{\textcolor[rgb]{0.31,0.60,0.02}{#1}}
\newcommand{\VariableTok}[1]{\textcolor[rgb]{0.00,0.00,0.00}{#1}}
\newcommand{\VerbatimStringTok}[1]{\textcolor[rgb]{0.31,0.60,0.02}{#1}}
\newcommand{\WarningTok}[1]{\textcolor[rgb]{0.56,0.35,0.01}{\textbf{\textit{#1}}}}
\usepackage{longtable,booktabs,array}
\usepackage{calc} % for calculating minipage widths
% Correct order of tables after \paragraph or \subparagraph
\usepackage{etoolbox}
\makeatletter
\patchcmd\longtable{\par}{\if@noskipsec\mbox{}\fi\par}{}{}
\makeatother
% Allow footnotes in longtable head/foot
\IfFileExists{footnotehyper.sty}{\usepackage{footnotehyper}}{\usepackage{footnote}}
\makesavenoteenv{longtable}
\usepackage{graphicx}
\makeatletter
\def\maxwidth{\ifdim\Gin@nat@width>\linewidth\linewidth\else\Gin@nat@width\fi}
\def\maxheight{\ifdim\Gin@nat@height>\textheight\textheight\else\Gin@nat@height\fi}
\makeatother
% Scale images if necessary, so that they will not overflow the page
% margins by default, and it is still possible to overwrite the defaults
% using explicit options in \includegraphics[width, height, ...]{}
\setkeys{Gin}{width=\maxwidth,height=\maxheight,keepaspectratio}
% Set default figure placement to htbp
\makeatletter
\def\fps@figure{htbp}
\makeatother
\setlength{\emergencystretch}{3em} % prevent overfull lines
\providecommand{\tightlist}{%
  \setlength{\itemsep}{0pt}\setlength{\parskip}{0pt}}
\setcounter{secnumdepth}{5}
\usepackage{booktabs}
\usepackage{amsthm, amsmath}
\makeatletter
\def\thm@space@setup{%
  \thm@preskip=8pt plus 2pt minus 4pt
  \thm@postskip=\thm@preskip
}
\makeatother
\DeclareMathOperator*{\End}{End}
\ifLuaTeX
  \usepackage{selnolig}  % disable illegal ligatures
\fi
\usepackage[]{natbib}
\bibliographystyle{apalike}

\title{An Introduction to Modular Representation Theory}
\author{Amritanshu Prasad}
\date{2022-06-30}

\usepackage{amsthm}
\newtheorem{theorem}{Theorem}[chapter]
\newtheorem{lemma}{Lemma}[chapter]
\newtheorem{corollary}{Corollary}[chapter]
\newtheorem{proposition}{Proposition}[chapter]
\newtheorem{conjecture}{Conjecture}[chapter]
\theoremstyle{definition}
\newtheorem{definition}{Definition}[chapter]
\theoremstyle{definition}
\newtheorem{example}{Example}[chapter]
\theoremstyle{definition}
\newtheorem{exercise}{Exercise}[chapter]
\theoremstyle{definition}
\newtheorem{hypothesis}{Hypothesis}[chapter]
\theoremstyle{remark}
\newtheorem*{remark}{Remark}
\newtheorem*{solution}{Solution}
\begin{document}
\maketitle

{
\setcounter{tocdepth}{1}
\tableofcontents
}
\hypertarget{semisimplicity-and-the-radical}{%
\chapter{Semisimplicity and the Radical}\label{semisimplicity-and-the-radical}}

\(\DeclareMathOperator*{\End}{End}\)
\(\DeclareMathOperator*{\rad}{rad}\)

\begin{exercise}
Determine when the ring \(\mathbf{F}_p[x]/(x^n-1)\) is semisimple.
Take \(n\) to be any positive integer and \(p\) to be a prime number.
Here \(\mathbf{F}_p\) denotes a finite field with \(p\) elements \citet{MR1715145}.
\end{exercise}

Let \(K\) be an algebraically closed field, and let \(A\) be a finite dimensional \(K\)-algebra.

\begin{lemma}
\protect\hypertarget{lem:simple}{}\label{lem:simple}Every simple \(A\)-module is a quotient of \({}_AA\).
\end{lemma}

\begin{proof}
Suppose that \(M\) is a simple \(A\)-module.
Pick any \(m\in M\) such that \(m\neq 0\).
Define \(\phi_m:A\to M\) by \(\phi_m(a)=am\).
Therefore \(\phi_m(A)\) is a non-trivial submodule of \(M\).
Since \(M\) is simple, \(\phi_m\) must be surjective.
Therefore, \(M\cong A/\mathrm{ker}\phi_m\).
\end{proof}

\begin{corollary}
Every simple module is isomorphic to the quotient of the regular module by a left ideal.
\end{corollary}

\begin{definition}
A left (right, or two-sided) ideal \(N\) of \(A\) is said to be \emph{nilpotent} if there exists a positive integer \(k\) such that \(N^k=0\).
Here \(N^k\) denotes the span of products of \(k\) elements in \(N\).
\end{definition}

\begin{lemma}
\protect\hypertarget{lem:nilker}{}\label{lem:nilker}Let \(M\) be a simple modules, \(m\in M\) be a non-zero elements, and let \(\phi_m\) be as in the proof of Lemma \ref{lem:simple}.
Then every nilpotent left ideal of \(A\) is contained in the kernel of \(\phi_m\).
\end{lemma}

Suppose that \(N\) is a nilpotent left ideal of \(A\).
Then for any \(n\in N\).
Then \(\phi_m(N)=Nm\).
If \(\phi_m(N)\) is non-zero, the \(Nm=M\).
In other words, there exists \(n\in N\) such that \(nm=m\).
It follows that \(n^km=m\) for all \(k\geq 0\).
However, if \(k\) is chosen such that \(N^k=0\), then \(n^k=0\), so \(m=n^km=0\), a contradiction.

\begin{lemma}
\protect\hypertarget{lem:sumnil}{}\label{lem:sumnil}If \(N_1\) and \(N_2\) are nilpotent left ideals, then \(N_1+N_2\) is also a nilpotent left ideal.
\end{lemma}

Choose \(k\) such that \(N_1^k=N_2^k=0\).
An element of \((N_1+N_2)^{2k}\) is a sum of products of \(2k\) elements that are drawn either from \(N_1\), or from \(N_2\).
Such a product will have at least \(k\) terms from \(N_1\) or at least \(k\) terms from \(N_2\).
Therefore such a product will lie in \(N_1^k\) of \(N_2^k\), hence will be \(0\).

\begin{exercise}
Let \(n_1=\begin{pmatrix}0&1\\0&0\end{pmatrix}\) and \(n_2=\begin{pmatrix}0&0\\1&0\end{pmatrix}\).
Then \(n_1\) and \(n_2\) are nilpotent, but \(n_1+n_2\) is not nilpotent.
Does this contradict Lemma \ref{lem:sumnil}?
\end{exercise}

\begin{corollary}
The algebra \(A\) has a unique maximal nilpotent left ideal.
\end{corollary}

\begin{proof}
Begin with any nilpotent left ideal \(N\) of \(A\) (for example, \(N=(0)\)).
If \(N\) is not maximal, then there exists another nilpotent left ideal \(N_2\) of \(A\) that is not contained in \(N_1\).
By Lemma \ref{lem:sumnil}, \(N_1+N_2\) is a nilpotent left ideal of \(A\) that is strictly larger than \(N_1\).
Since \(A\) is finite dimensional, this process must terminate in a finite number of steps, thereby giving rise to a maximal nilpotent left ideal of \(A\).
If there was more than one maximal nilptent left ideal in \(A\), the sum of two such would be a nilpotent ideal strictly larger than either of them, contradicting their maximality.
\end{proof}

\begin{definition}
The \emph{radical} of \(A\), denoted \(\rad(A)\) is the maximal nilpotent left ideal of \(A\).
\end{definition}

\begin{proposition}
The ideal \(\rad(A)\) is a two-sided ideal.
\end{proposition}

\begin{proof}
Note that \(\rad(A)A\) is a two-sided ideal.
Also, if \(\rad(A)^k=0\), then
\[
[\rad(A)A]^k = \rad(A)[A\rad(A)]^{k-1}A = \rad(A)^kA = 0.
\]
Therefore \(\rad(A)A\) is a nilpotent left ideal containing \(\rad(A)\), hence is equal to \(\rad(A)\).
\end{proof}

\begin{theorem}
\(A\) is semisimple if and only if \(\rad(A)=0\).
\end{theorem}

\begin{proof}
Suppose that \(A\) is semisimple.
Then, as a left \(A\)-module,
\[
A = M_1\oplus \dotsb \oplus M_k
\]
for simple \(A\)-modules \(M_1,\dotsc,M_k\).
Suppose that, under this decomposition the identity element of \(A\) is written as
\begin{equation}
\label{eq:pou}
1 = \epsilon_1 + \dotsb + \epsilon_k,
\end{equation}
where \(\epsilon_i\in M_i\).
By Lemma \ref{lem:nilker}, every nilpotent left ideal of \(A\) is contained in \(\ker \phi_{\epsilon_i}\).
Therefore every nilpotent left ideal of \(A\) is contained in \(\ker(\phi_{\epsilon_1}+\dotsb + \phi_{\epsilon_k})\).
But \eqref{eq:pou} implies that
\[
\phi_{\epsilon_1}+\dotsb + \phi_{\epsilon_k} = \phi_1 = \mathrm{id}_A.
\]
If follows that \(A\) has no nontrivial nilpotent left ideals, and hence \(\rad(A)=0\).

For the converse, suppose that \(\rad(A)=0\).
If \({}_AA\) is simple, then \(A\) is semisimple.
If not, there \(A\) contains non-trivial proper left ideals.
Let \(N\) be a minimal non-trivial proper left ideal of \(A\).
Since \(N\) is not nilpotent, \(N^2\neq (0)\).
Therefore \(N^2=N\).
It follows that there exists \(a\in N\) such that \(Na\neq 0\).
Again, minimality of \(N\) implies that \(Na=N\).

Let
\[
B = \{b\in N\mid ba\neq 0\}.
\]
Then \(B\) is a left ideal of \(N\).
\end{proof}

\begin{exercise}
If \(N\subset A\) is a left ideal, \(N\) is a submodule of \({}_AA\).
Thus, the vector space \(N/A\) inherits the structure of a left \(A\)-module.
Show that if \(N\) is a two-sided ideal, then \(N/A\) inherits the structure of a \(K\)-algebra.
In particular \(A/\rad(A)\) is a \(K\)-algebra.
\end{exercise}

\hypertarget{intro}{%
\chapter{The Radical of the Group Algebra}\label{intro}}

You can label chapter and section titles using \texttt{\{\#label\}} after them, e.g., we can reference Chapter \ref{intro}. If you do not manually label them, there will be automatic labels anyway, e.g., Chapter \ref{methods}.

Figures and tables with captions will be placed in \texttt{figure} and \texttt{table} environments, respectively.

\begin{Shaded}
\begin{Highlighting}[]
\FunctionTok{par}\NormalTok{(}\AttributeTok{mar =} \FunctionTok{c}\NormalTok{(}\DecValTok{4}\NormalTok{, }\DecValTok{4}\NormalTok{, .}\DecValTok{1}\NormalTok{, .}\DecValTok{1}\NormalTok{))}
\FunctionTok{plot}\NormalTok{(pressure, }\AttributeTok{type =} \StringTok{\textquotesingle{}b\textquotesingle{}}\NormalTok{, }\AttributeTok{pch =} \DecValTok{19}\NormalTok{)}
\end{Highlighting}
\end{Shaded}

\begin{figure}

{\centering \includegraphics[width=0.8\linewidth]{bookdown-demo_files/figure-latex/nice-fig-1} 

}

\caption{Here is a nice figure!}\label{fig:nice-fig}
\end{figure}

Reference a figure by its code chunk label with the \texttt{fig:} prefix, e.g., see Figure \ref{fig:nice-fig}. Similarly, you can reference tables generated from \texttt{knitr::kable()}, e.g., see Table \ref{tab:nice-tab}.

\begin{Shaded}
\begin{Highlighting}[]
\NormalTok{knitr}\SpecialCharTok{::}\FunctionTok{kable}\NormalTok{(}
  \FunctionTok{head}\NormalTok{(iris, }\DecValTok{20}\NormalTok{), }\AttributeTok{caption =} \StringTok{\textquotesingle{}Here is a nice table!\textquotesingle{}}\NormalTok{,}
  \AttributeTok{booktabs =} \ConstantTok{TRUE}
\NormalTok{)}
\end{Highlighting}
\end{Shaded}

\begin{table}

\caption{\label{tab:nice-tab}Here is a nice table!}
\centering
\begin{tabular}[t]{rrrrl}
\toprule
Sepal.Length & Sepal.Width & Petal.Length & Petal.Width & Species\\
\midrule
5.1 & 3.5 & 1.4 & 0.2 & setosa\\
4.9 & 3.0 & 1.4 & 0.2 & setosa\\
4.7 & 3.2 & 1.3 & 0.2 & setosa\\
4.6 & 3.1 & 1.5 & 0.2 & setosa\\
5.0 & 3.6 & 1.4 & 0.2 & setosa\\
\addlinespace
5.4 & 3.9 & 1.7 & 0.4 & setosa\\
4.6 & 3.4 & 1.4 & 0.3 & setosa\\
5.0 & 3.4 & 1.5 & 0.2 & setosa\\
4.4 & 2.9 & 1.4 & 0.2 & setosa\\
4.9 & 3.1 & 1.5 & 0.1 & setosa\\
\addlinespace
5.4 & 3.7 & 1.5 & 0.2 & setosa\\
4.8 & 3.4 & 1.6 & 0.2 & setosa\\
4.8 & 3.0 & 1.4 & 0.1 & setosa\\
4.3 & 3.0 & 1.1 & 0.1 & setosa\\
5.8 & 4.0 & 1.2 & 0.2 & setosa\\
\addlinespace
5.7 & 4.4 & 1.5 & 0.4 & setosa\\
5.4 & 3.9 & 1.3 & 0.4 & setosa\\
5.1 & 3.5 & 1.4 & 0.3 & setosa\\
5.7 & 3.8 & 1.7 & 0.3 & setosa\\
5.1 & 3.8 & 1.5 & 0.3 & setosa\\
\bottomrule
\end{tabular}
\end{table}

You can write citations, too. For example, we are using the \textbf{bookdown} package \citep{R-bookdown} in this sample book, which was built on top of R Markdown and \textbf{knitr} \citep{xie2015}.

\hypertarget{literature}{%
\chapter{Literature}\label{literature}}

Here is a review of existing methods.

\hypertarget{methods}{%
\chapter{Methods}\label{methods}}

We describe our methods in this chapter.

Math can be added in body using usual syntax like this

\hypertarget{math-example}{%
\section{math example}\label{math-example}}

\(p\) is unknown but expected to be around 1/3. Standard error will be approximated

\[
SE = \sqrt(\frac{p(1-p)}{n}) \approx \sqrt{\frac{1/3 (1 - 1/3)} {300}} = 0.027
\]

You can also use math in footnotes like this\footnote{where we mention \(p = \frac{a}{b}\)}.

We will approximate standard error to 0.027\footnote{\(p\) is unknown but expected to be around 1/3. Standard error will be approximated

  \[
  SE = \sqrt(\frac{p(1-p)}{n}) \approx \sqrt{\frac{1/3 (1 - 1/3)} {300}} = 0.027
  \]}

\hypertarget{applications}{%
\chapter{Applications}\label{applications}}

Some \emph{significant} applications are demonstrated in this chapter.

\hypertarget{example-one}{%
\section{Example one}\label{example-one}}

\hypertarget{example-two}{%
\section{Example two}\label{example-two}}

\hypertarget{final-words}{%
\chapter{Final Words}\label{final-words}}

We have finished a nice book.

  \bibliography{book.bib,packages.bib}

\end{document}
